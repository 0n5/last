\chapter*{General Introduction}
\label{Intro}
\addcontentsline{toc}{chapter}{General Introduction}
\markboth{General Introduction}{General Introduction}
The reliance on technology nowadays for the public is implanted in the society. More activities and services are converting to the web each day that very few now can be performed without the use of digital devices. Most individuals and organizations have an electronic presence, using it to deliver services, communicate, plan meetings, play, socialize, and study. This advancement in technology means that most data is being digitized and devices are connected all over the world, making the spreading of data easier than before.\\
Digital devices like cell and smart phones, computers, and tablets store a lot of information relative to it's owner which makes it pretty inconvenient if it was stolen or hacked. These devices can be attacked as much as they can be part of a crime, be it a cybercrime or not. Considering the risks involved, companies have been investing a lot, these last few years, in the security of their information systems by seeking penetration testing services and implementing monitoring platforms like IDS, firewalls, and anti-viruses.\\
With the evolve of these technologies, engineers have to update and change their systems and tweak the products they implement per the company's needs, potentially leaving behind vulnerabilities that could be exploited. As no system is 100\% secure, incidents still occur, and organizations get hacked for various reasons.\\
The need for incident response and digital forensics has emerged, when a huge number of modern crimes where committed using computers and evidence was stored in there. Breached companies now hire professional investigators through consulting firms to assess the situation. A digital forensics investigation's goal is to determine the nature and events concerning a crime and locate the perpetrator by following an organized investigative procedure. The questions that drive a DF investigation are what happened, who did it, how and why it happened, and consulting firm like EY offer such security related services.\\
This project was implemented within the EY Advanced Cyber Security team. This team is a group of ethical hackers that offer consulting services to clients regarding information security. The team cover a variety of services such as external and internal penetration testing, code and configuration review, security audit, Red team services, and also Blue team services which contain precisely digital forensics investigation.\\
Facing the fact that modern computers have high specifications that require harder and longer processing, Data can get very large and can spread across devices, and malicious activities and programs are evolving in a way that they leave lesser trace by using volatile memory devices such as RAM rather than a hard disk. This means that the traditional computer forensic techniques and tools employed to an analyse a single device are no longer as effective. Investigations are even taking a move into using techniques like triage for the analysis instead of a full investigation of the E-evidence's data. This can lead into false negatives due to the human's fatigue and boredom when repeating such long tasks in the same way.\\
Therefore, a need to automate and simplify the process of forensic investigation is under the scope. This can be achieved by regrouping the traditional open-source tools into one platform and automating digital evidence processing, data extraction, and analysis. With such volumes of data, reduction of the data to be inspected into pre-selected categories is also important to keep a low false positives rate and rely on the investigator's judgmental decision as incident response should be no subject of doubt.\\
Correspondingly, this project is carried out as part of the preparation for the end-of-studies project presented for the National Diploma in Computing Science and Technologies in Network and Services Administration at the higher institute of computer science «ISI».\\
The project's objective is to achieve the development of the nonexistent platform for digital forensics regrouping tools and automating phases of various forensic analysis types for the investigation process as described above.\\
This report is organized as follows:
\begin{itemize}
    \item The first chapter introduces the host organization, in which the project has taken place. Then gives an overview on cybercrime and digital forensics methodology. And finally an analysis of existent solutions and our proposed solution.
    \item The second chapter presents the design phase of the project, identifying the specifications and introducing a set of diagrams.
    \item The third chapter describes the realization and implementation phase of the project, including the platform's architecture and tools used.
    \item The fourth chapter presents a case study using a real case scenario and the developed platform.
    \item Finally, we end our report with a general conclusion that summarizes the work we have accomplished and presents our outlook
\end{itemize}