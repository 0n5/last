\chapter*{Conclusion and Outlook}
\label{Intro}
\addcontentsline{toc}{chapter}{Conclusion and Outlook}
\markboth{Conclusion and Outlook}{Conclusion and Outlook}

Today, news of security breaches dominate headlines on weekly basis. Hackers are learning to hide their identity and traces. On the victim's part, it is important to investigate the incident carefully since hackers get away with it and are nearly unidentified most of the time. Also, With the increase usage of new technologies and computers, most crimes are either cybercrimes or at least have evidence on a digital device. Due to the excessive data submitted for analysis, it is no more appropriate or easy to examine every single bit of the exhibited data single-handedly with specific tools to use for each bit of it. Therefore, modern solutions improving and automating the investigation process are \vspace{3mm}required.\\
Our end of studies project, conducted within the Advanced Cyber Security team of AMC Ernst \& Young, aims to design, develop and integrate a secured web platform that automates the examination, analysis, and presentation of digital media and data, in digital forensics \vspace{3mm}investigations.\\
This work focused on automating some of the digital forensics process, while including open-source tools based on their utility and effectiveness, and automating the usage of these tools to extract and present their outputs in a clear manner. Also, as a DF lab now requires a huge budget to get every tool out there, because each one focuses on a single media type, our focus was on including various types of digital forensics in one platform which were divided into three categories: \vspace{3mm}memory analysis, network analysis, and disk analysis.\\
We started our work by performing a literature review on cyber crimes and digital forensics types, steps, and techniques, and also a research on existing solutions and DFIR tools in order to identify their weaknesses and limits, and adapt our solution to a better version. We then were able to specify the functional and non-functional requirements of our platform and move to the design phase where we modeled our solution using use case and sequence diagrams to better describe its functions. We then began with the implementation of our proposed solution using python scripts, for automating the analysis and usage of the used tools based on the type of analysis for the selected case, in the back-end and Django application as our front-end. Finally, we ended up with testing our platform using digital evidence from manually reproduced real case \vspace{3mm}scenarios.\\
The accomplishment of this project allowed me to integrate into the professional field and to apply the technical and theoretical skills I acquired during my three years of studies at the Higher Institute of Computer Science of El Manar university. I have also discovered new concepts that revolve around ethical hacking, read team assessment, digital forensics and incident response techniques and technologies. This internship also gave me the opportunity to interact with cyber security professionals and get their feedback and guidance, which allowed me to develop my communication skills, ability to work as a team, and undergo a real world and professional \vspace{3mm}experience.\\
This solution can be further more improved by adding more categories of digital forensics analysis, like log forensics, and more tools for each categories, and fully automating the extraction and presentation phases. We can also extend this work by implementing an automated decision maker based on artificial intelligence techniques, which are not favorable for lacking the ability to conduct DF advanced analysis, but would be useful in detecting and linking notable data at first. Also, the platform supports multiple operating systems forensics, but is proven to be totally functional only for Windows, which we intend to improve by taking into consideration the difference between operating systems architectures in the back-end evidence manipulation.